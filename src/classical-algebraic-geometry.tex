
\chapter{Algebraic varieties}

We collect some basic results from the first chapter of
\cite{görtz2010algebraic}.

\section{Zariski topology and algebraic sets}


\begin{definition}
  \label{def:vanishing-sets}
  Consider a subset $M\subset \field{k}[T_1,\cdots ,T_n ] \eqqcolon
  \field{k}[\underline{T}]$. We call define the \textit{vanishing set}, or set
  of common zeros, of the polynomials in $M$ by
  \begin{align*}
    V(M)\coloneqq \{(t_1,\cdots, t_n)\in \field{k}^n\mid \forall f\in
    M \; ; \; f(t_1,\cdots, t_n)=0\}.
  \end{align*}
\end{definition}


\begin{definition}
  \label{def:zariski-topology}
  The sets $V(\mathfrak{a})$ where $\mathfrak{a}$ runs through the set
  of ideals of $\field{k}[\underline{T}]$, are the closed sets of a
  topology on $\field{k}^n$, called the Zariski topology.
\end{definition}

\begin{exercise}
  Prove that this is a well defined topology, or alternatively read
  the proof in \cite[Prop. 1.3]{görtz2010algebraic}.
\end{exercise}

\begin{definition}
  \label{def:algebraic-space}
  We call the topological space $(\field{k}^n, \{V(\mathfrak{a})\}_{\mathfrak{a}
    \; \text{is ideal}})$ of the space $\field{k}^n$ with the Zariski
  topology the \textit{affine space of dimension $n$} and denote it
  with $\field{A}^n(\field{k})$.

  Closed subspaces of $\field{A}^n(\field{k})$ are called
  \textit{affine algebraic sets.}
\end{definition}

Sets consisting of one point $x =(x_1,\cdots ,x_n) \in
\field{A}^n(\field{k})$ are closed because $\{x\} = V
(\mathfrak{m}_x)$, where $\mathfrak{m}_x =(T_1−x_1,\cdots ,T_n−x_n)$ is the kernel of the evaluation homomorphism $\field{k}[\underline{T} ] \rightarrow \field{k}$ which sends $f$ to $f (x)$. As finite unions of closed sets are again closed, we see that all finite subsets of $\field{A}^n(\field{k})$ are close.

\begin{definition}
  \label{def:radical-of-an-ideal}
  Let $\ring{A}$ be a ring. For an ideal $\mathfrak{a} \subset
  \ring{A}$ we call $\rad \mathfrak{a} \coloneqq \{f\in \ring{A} \mid
  \exists r\ in \naturals_0 \text{ such that } f^r \in \mathfrak{a}\}$
  the \textit{radical} of a.

  If $\rad\mathfrak{a} = \mathfrak{a}$ we call $\mathfrak{a}$ a
  \textit{radical ideal}.
\end{definition}

\begin{proposition}
  \label{prop:properties-of-the-radical}
  Let $\ring{A}$ be a ring and $\mathfrak{a}$ an ideal.
  \begin{enumerate}[label=(\roman*)]
  \item We have $\rad \mathfrak{a} = \rad (\rad \mathfrak{a})$.
  \item If $\ring{A}$ is finitely generated as a $\field{k}$ algebra
    for a field (not necessarily algebraically closed), we have
    \begin{align*}
      \rad \mathfrak{a}
      =
      \bigcap_{\substack{\mathfrak{a}\subset
      \mathfrak{p}\subset \ring{A}\\
      \mathfrak{p}\text{ prime ideal}}}
      \mathfrak{p}
      =
      \bigcap_{\substack{\mathfrak{a}\subset
      \mathfrak{m}\subset \ring{A}\\
      \mathfrak{m}\text{ maximal ideal}}}
      \mathfrak{m}
    \end{align*}
  \item We have $V(\rad{\mathfrak{a}})=V(\mathfrak{a})$.
  \item The ring $\ring{A}/\mathfrak{a}$ is \textit{reduced}, i.e.\ it
    contains no non-zero idempotent elements.
  \item Every prime ideal is radical.
  \end{enumerate}
\end{proposition}

\begin{definition}
  \label{def:vanishing-ideal}
  For $Z\subset \field{A}^n(\field{k})$ we define the vanishing ideal
  of the subset $Z$ by
  \begin{align*}
    I(Z) \coloneqq \{f\in \field{k}[\underline{T}]\mid \forall x\in Z
    \text{ we have } f(x) = 0\}\subset \field{k}[\underline{T}].
  \end{align*}  
\end{definition}

\begin{proposition}
  \label{prop:vanishing-ideals-and-vanishing-sets-in-affine-space}
  Consider the affine space $\field{A}^n (\field{k})$. We have the
  following results:
  \begin{enumerate}[label=(\roman*)]
  \item For a subset $Z$ we have
    \begin{align*}
      I(Z)= \bigcap_{x\in Z} \mathfrak{m}_x.
    \end{align*}
  \item For an ideal $\mathfrak{a}\subset \field{k}[\underline{T}]$ we
    have
    \begin{align*}
      I(V(\mathfrak{a})) = \rad \mathfrak{a}
    \end{align*}
  \item For a subset $Z\subset \field{A}^n(\field{k})$ we have
    \begin{align*}
      V(I(Z)) = \overline{Z}
    \end{align*}
  \end{enumerate}
\end{proposition}

\begin{corollary}
  \label{cor:correspondence-ideals-and-subsets}
  The maps
  \begin{equation}
    \begin{tikzcd}[nodes in empty cells, column sep= 2cm, row sep=1.5cm]
      \{\text{radical ideals }\mathfrak{a}\text{ of
      }\field{k}[\underline{T}]\} \ar[r,shift left=1, "\mathfrak{a}\mapsto
      V(\mathfrak{a})"]
      &
      \{ \text{closed subsets } Z \text{ of }
      \field{A}^n(\field{k}) \} \ar[l,shift left=1," I(Z)\mapsfrom Z"]
    \end{tikzcd}
  \end{equation}
  are mutually inverse bijections, whose restrictions define a
  bijection
    \begin{equation}
    \begin{tikzcd}[nodes in empty cells, column sep= 2cm, row sep=1.5cm]
      \{
      \text{maximal ideals of } \field{k}[\underline{T}]
      \}
      \ar[r,leftrightarrow]
      &
      \{
      \text{points of }\field{A}^n(\field{k})
      \}
    \end{tikzcd}.
  \end{equation}
\end{corollary}

\section{Irreducible topological  spaces}

\begin{definition}
  \label{def:irreducible-topological-spaces}
  A non-empty topological space $X$ is called \textit{irreducible} if
  $X$ cannot be expressed as the union of two proper closed subsets.

  A non-empty subset $Z$ of $X$ is called \textit{irreducible} if $Z$ is irreducible when we endow it with the induced topology.
\end{definition}

\begin{proposition}
  \label{prop:characterization-of-irreducible-topological-spaces}
  Let $X$ be a non-empty topological space. The following assertions
  are equivalent.
  \begin{enumerate}[label=(\roman*)]
  \item $X$ is irreducible.
  \item Any two non-empty open subsets of $X$ have a non-empty intersection.
  \item Every non-empty open subset is dense in $X$.
  \item Every non-empty open subset is connected.
  \item Every non-empty open subset is irreducible.
  \end{enumerate}
\end{proposition}

\begin{corollary}
  \label{cor:maps-from-irreducible-sets}
  Let $f \colon X \rightarrow Y$ be a continuous map of topological spaces. If $Z \subset X$ is an irreducible subspace, its image $f (Z)$ is irreducible.
\end{corollary}

%%% Local Variables:
%%% mode: latex
%%% TeX-master: "main"
%%% End:
