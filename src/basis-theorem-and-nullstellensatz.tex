\chapter{Some basic commutative algebra}

This small chapter serves as a short collection on the basis theorem
and the Nullstellensatz by Hilbert. This is also a collection of some
commutative algebra needed to define affine varieties.


\section{Hilbert basis theorem}

\begin{definition}
  \label{def:noetherian-ring-noetherian-module}
  A ring $\ring{A}$ is said to be Noetherian if it satisfies the
  following three equivalent conditions:
  \begin{enumerate}[label=(\roman*)]
  \item Every non-empty set of ideals in $\ring{A}$ has a maximal
    element.
  \item Every ascending chain of ideals in $\ring{A}$ is stationary.
  \item Every ideal in $\ring{A}$ is finitely generated.
  \end{enumerate}
  Similarly for a $\ring{R}$-module $M$ we call $M$ Noetherian if it
  satisfies the following three equivalent conditions:
  \begin{enumerate}[label=(\roman*)]
  \item Every non-empty set $T$ of submodules of $M$ has a maximal
    element.
  \item Every ascending chain of submodules of $M$ is stationary.
  \item Every submodule $N\subset M$ is finitely generated.
  \end{enumerate}
\end{definition}

To show the equivalence of the first two properties for rings and
modules we can look at the following general result on partially
ordered sets.

\begin{proposition}
  \label{prop:equivalence-noetherian-conditions-rings}
  The following conditions on a partially ordered set $\mathcal{P}$
  are equivalent:
  \begin{enumerate}[label=(\roman*)]
  \item\label{item:inc-secquence} Every increasing sequence $x_1 \leq x_2 \leq \cdots$ in
    $\mathcal{P}$ is stationary, i.e.\ there is an index $n\in
    \naturals$ such that for all $m\geq n$ we have $x_m = x_n$.
  \item\label{item:max-element} Every non-empty subset of
    $\mathcal{P}$ has a maximal element.
  \end{enumerate}
\end{proposition}

\begin{proof}
  \autoref{prop:equivalence-noetherian-conditions-rings}-\ref{item:inc-secquence}
  $\Rightarrow$
  \autoref{prop:equivalence-noetherian-conditions-rings}-~\ref{item:max-element}
  Assume that a non-empty subset $T\subset \mathcal{P}$ and that it
  contains no maximal element. Then we can inductively construct an
  ascending sequence $(x_i)_{i\in \naturals}$ such that $x_i\leq
  x_{i+1}$ which is not stationary, since this would contradict the
  existence of a maximal element.

  \autoref{prop:equivalence-noetherian-conditions-rings}-~\ref{item:max-element}
  $\Rightarrow$
  \autoref{prop:equivalence-noetherian-conditions-rings}-~\ref{item:inc-secquence}
  For an ascending sequence $(x_i)_{\i\in \naturals}$ this set has a
  maximal element. This means that the sequence has to stabilize.
\end{proof}

Now to show that the third property is also equivalent to the first
two properties we just need to show that every submodule is finitely
generated, as an ideal is just a special case of a submodule of the
ring $\ring{A}$ considered as a module over itself.

\begin{proposition}
  \label{prop:submodules-of-noetherian-module-finitely-gernerated}
  A module $M$ over a ring $\ring{A}$ is Noetherian if and only if
  every submodule $N\subset M$ is finitely generated.
\end{proposition}

\begin{proof}
  Lets first consider the ``if'' direction. Consider a submodule $N
  \subset M$ and consider the set $\mathcal{P}$ of all finitely
  generated submodules of
  $N$ ordered by inclusion. The set $\mathcal{P}$ is non-empty since
  $\{0\}\in \mathcal{P}$ and thus $\mathcal{P}$ has a maximal element
  $N_0$. If $N\neq N_0$ consider the submodule $N_0\oplus \ring{A}x$
  where $x\in N$, $x\notin N_0$. This is still finitely generated and
  strictly contains $N_0$, which contradicts the maximality. Thus
  $N=N_0$ and $N$ is finitely generated.

  For the ``only if'' direction consider an ascending chain of
  submodules $T\coloneqq (M_n)_{n\in \naturals} $ of $M$. Then the module
  $\cup_{n\in \naturals} M_n$ is finitely generated by
  assumption. Thus $T = \ring{A}\langle x_1, \cdots , x_r\rangle$ with
  $x_i \in M_{n_i}$ and
  set $n=\max_{i=1}^r n_i$. Then each $x_i\in M_n$ and thus $M_n=M$
  which means that the chain is stationary. 
\end{proof}

\begin{proposition}
  \label{prop:middle-in-sec-noetherian-equiv-all-noetherian}
  Consider a short exact sequence
  \begin{equation}
    \label{eq:sec-middle-noetherian-equiv-all-noetherian}
    \begin{tikzcd}[nodes in empty cells, column sep= 1.5cm, row sep= 1.5cm]
      0 \ar[r] & M' \ar[r,"\alpha"] & M \ar[r, "\beta"] & M'' \ar[r] & 0
    \end{tikzcd}
  \end{equation}
  then $M$ is Noetherian if and only if $M'$ and $M''$ are Noetherian.
\end{proposition}

\begin{proof}
  Consider an ascending chain of submodules $(M_n)_{n\in \naturals}$
  in $M$. This gives rise to commutative diagrams of the form
  \begin{equation}
    \label{eq:diagram-for-five-lemma-proof-of-sec-of-noetherian-modules}
    \begin{tikzcd}[nodes in empty cells, column sep= 1.5cm, row sep=1.5cm]
      0 \ar[r]
      & \alpha^{-1}(M_n) \ar[r,"\alpha"] \ar[d, hook]
      & M_n \ar[r, "\beta"] \ar[d, hook]
      & \beta(M_n) \ar[r] \ar[d, hook]
      & 0
      \\
      0 \ar[r] 
      & \alpha^{-1}(M_{n+1}) \ar[r,"\alpha"]
      & M_{n+1} \ar[r, "\beta"]
      & \beta(M_{n+1}) \ar[r]
      & 0
    \end{tikzcd}.
  \end{equation}
  In this diagram the rows are exact by construction and since
  \eqref{eq:sec-middle-noetherian-equiv-all-noetherian} is exact.
  Now assume that $M$ is Noetherian. Then we can pick $n\in \naturals$
  big enough such that the middle inclusion is an isomorphism. Then
  the two other inclusions are isomorphisms as well by the five
  lemma. Conversely, if $M'$ and $M''$ are Noetherian, we can pick $n$
  big enough such that the chains in $M'$ and $M''$ have
  stabilized. Then, again by the five lemma, the middle inclusion is
  also an isomorphism.
\end{proof}

Now an easy corollary of this is, that direct sums of Noetherian
modules are again Noetherian.

\begin{corollary}
  \label{cor:direct-sum-of-noetherian-is-noetherian}
  Let $M_i$, $i\in \{1,\cdots, n\}$ be Noetherian modules. Then
  $\bigoplus_{i=1}^n M_i$ is also Noetherian.
\end{corollary}

\begin{proof}
  This is an immediate consequence of
  \autoref{prop:middle-in-sec-noetherian-equiv-all-noetherian} since
  the sequence
  \begin{equation}
    \begin{tikzcd}[nodes in empty cells, column sep= 1.5cm, row sep=1.5cm]
      0 \ar[r]
      & M_k \ar[r]
      & \bigoplus_{i=1}^k M_i \ar[r]
      & \bigoplus_{i=1}^{k-1}M_i \ar[r]
      & 0
    \end{tikzcd}
  \end{equation}
  is exact, the statement follows by induction.
\end{proof}

Another nice feature of finitely generated modules is, that they are
quotients of $\ring{A}^n$ for some $n\in \naturals$.

\begin{proposition}
 \label{prop:finite-module-is-quotient-of-free-module}
 Let $M$ be a $\ring{A}$-module. Then $M$ is finitely generated if and
 only if it is a quotient of $\ring{A}^n$ for some $n\in \naturals$.
\end{proposition}

\begin{proof}
  Let $M = \ring{A}\langle x_1,\cdots ,x_n\rangle$ be finitely
  generated. Then we define a $\ring{A}$-module morphism by defining
  \begin{align*}
    \phi \colon \ring{A}^n \rightarrow M, \quad \phi(a_1,\cdots, a_n)
    \coloneqq a_1 x_1 + \cdots + a_n x_n.
  \end{align*}
  This map is obviously surjective. Conversely, if we have quotient
  map $\phi \colon \ring{A}^n \rightarrow M$, then $M$ is generated by
  $\phi(e_i)$, where $(e_i)_j = \delta_{i,j} 1$ for $i=\{1,\cdots, n\}$.
\end{proof}

And last but not least we can get a nice result, that all finitely
generated $\ring{A}$-modules over a Noetherian ring are also
Noetherian.

\begin{proposition} \label{prop:finite-modules-over-noetherian-are-noetherian}
  Let $M$ be a finitely generated $\ring{A}$-module and $\ring{A}$
  Noetherian. Then $M$ is Noetherian.
\end{proposition}

\begin{proof}
  This proof follows immediately from
  \autoref{prop:finite-module-is-quotient-of-free-module} since the
  corresponding quotient map fits into the short exact sequence
  \begin{equation}
    \begin{tikzcd}[nodes in empty cells, column sep= 1.5cm, row sep=1.5cm]
      0 \ar[r]
      & \ker (\phi) \ar[r, hook]
      & \ring{A}^n \ar[r, "\phi"]
      & M \ar[r]
      & 0
    \end{tikzcd}
  \end{equation}
  from which the statement follows by \autoref{prop:middle-in-sec-noetherian-equiv-all-noetherian}.
\end{proof}

Now we have all tools we need to prove Hilbert basis theorem.

\begin{theorem}[Hilbert basis theorem]
  \label{thm:hilbert-basis-thm}
  See~\cite{atiyah1994introduction}. Let $\ring{A}$ be a Noetherian ring, then the polynomial ring
  $\ring{A}[x]$ is Noetherian. 
\end{theorem}

\begin{proof}
  Consider an ideal $\mathfrak{a}\subset \ring{A}[x]$. We want to show
  that $\mathfrak{a}$ is finitely generated. Then, by
  \autoref{def:noetherian-ring-noetherian-module}, the ring
  $\ring{A}[x]$ would be Noetherian.

  We first consider the leading coefficients of polynomials $p$ in
  $\mathfrak{a}$ yield an ideal $I\subset \ring{A}$. Since $\ring{A}$
  is assumed to be Noetherian, this ideal is finitely generated, i.e.\
  $I = \ring{A}\langle a_1,\cdots,a_n \rangle$. Now we can pick the
  corresponding polynomials $f_i = a_i x^{r_i} + (\text{lower
    terms})$. For the ideal generated by these polynomials we get
  $\mathfrak{a}' \coloneqq \ring{A}\langle f_1, \cdots , f_n\rangle
  \subset \mathfrak{a}$.

  We want to show that $\mathfrak{a} = \mathfrak{a}'$. For this we pick
  some $f= ax^m + (\text{lower
    terms})$ for $a\in I$, i.e.\ $f\in \mathfrak{a}$.
  If $m\geq r$ for $r = \max_{i=1}^n r_i=\max_{i=1}^n \deg(f_i)$ we
  can write $a = \sum_{i=1}^n u_i a_i$ with $u_i\in\ring{A}$,
  $i=1,\cdots, n$ since $a\in I$ and $I $ is finitely generated. We
  can thus consider
  \begin{align*}
    f-\sum_{i=1}^nu_i f_i x^{m-r_i}.
  \end{align*}
  This is an element in $\mathfrak{a}$ and has degree $<m$.

  We can thus write $f= g + h$ for a polynomial $g$ with $\deg(g)< r$
  and $h\in \mathfrak{a}'$.

  Now consider $M= \ring{A}\langle 1,x,x^2,\cdots, x^{r-1}\rangle$, then we
  have shown that $\mathfrak{a}= (\mathfrak{a}\cap M) +
  \mathfrak{a}'$. Since $M$ is finitely generated as  a
  $\ring{A}$-module, it is quotient of $\ring{A}^n$
  which means that it is Noetherian as well by
  \autoref{prop:finite-modules-over-noetherian-are-noetherian}. Since
  $\mathfrak{a}\cup M $ is a submodule of $M$ it is thus also finitely
  generated. Finally if we have generators $\{f_i\}_{i=1,\cdots,n}$
  of $\mathfrak{a}'$ and generators $\{g_i\}_{i=1,\cdots, m}$ of
  $\mathfrak{a}\cup M $ they generate $\mathfrak{a}$ together.

  This shows that any ideal $\mathfrak{a}$ is finitely generated, even
  as a $\ring{A}$-module and hence $\ring{A}[x]$ is Noetherian.
\end{proof}

\begin{corollary} \label{cor:polynomials-in-multiple-variables-noetherian}
  For a Noetherian ring $\ring{A}$ the polynomials
  $\ring{A}[x_1,\cdots,x_n]$ is Noetherian.
\end{corollary}

\begin{proof}
  This is just an inductive application of Hilbert basis theorem on
  $\ring{A}[x,y]= (\ring{A}[x])[y]$.
\end{proof}


% \section{Noether normalization theorem}

% \begin{definition}[Ring extension]
%   \label{def:ring-extension}
%   Consider rings $\ring{A}$, $\ring{B}$.
%   Then we define the following:
%   \begin{enumerate}[label=(\roman*)]
%   \item If $\ring{A}\subset \ring{B}$ we call $\ring{B}$ an extension
%     ring of $\ring{A}$. We also call $\ring{A}\subset \ring{B}$ a ring
%     extension.
%   \item Let $\ring{A}\subset \ring{B}$ be a ring extension. We call
%     $b\in \ring{B}$ integral over $\ring{A}$ if there is a monic
%     polynomial $p\in \ring{A}[x]\subset \ring{B}[x]$ such that
%     $p(b)=0$.
%   \item For a ring extension $\ring{A}\subset \ring{B}$ we call
%     $\ring{B}$ integral over $\ring{A}$ if every $b\in \ring{A}$ is
%     integral over $\ring{A}$. In this case we say that
%     $\ring{A}\subset \ring{B}$ is an integral ring extension.
%   \item A morphism $f\colon \ring{A}\rightarrow \ring{B}$ is called
%     integral if $f(\ring{A})\subset \ring{B}$ is an
%     integral ring extension.
%   \item A ring extension $\ring{A}\subset \ring{B}$ is called finite
%     if $\ring{B}$ is finitely generated as a $\ring{A}$-module. 
%   \end{enumerate}
% \end{definition}

% \begin{lemma}
%   \label{lemma:multiindices-for-noether-normalization}
%   Given a vector $a=(a_1,\cdots,a_n)$ and a multiindex
%   $\beta = (\beta_1,\cdots,\beta_n)$ we set $a\cdot \beta =
%   \sum_{i=1}^n a_i \beta_i$.

%   Consider a finite set of multiindices
%   $I\coloneqq \{\beta_i\}_{i=1,\cdots,m}\subset \naturals_0^n$ and define
%   \begin{align*}
%     M_i =\max_{j=1,\cdots,n}(\beta_i)_j -
%     \min_{j=1,\cdots,n}(\beta_i)_j.
%   \end{align*}
%   If $a$ fulfills $a_{i-1}>\sum_{k=i}^n M_k a_k$ for $i=2,\cdots,n$
%   with $a_n \geq 0$, we have the following equivalence:
%   \begin{equation}
%     (a\cdot \beta = a \cdot \alpha) \; \Leftrightarrow \; (\alpha = \beta)
%   \end{equation}
%   for all $\beta, \alpha \in I$.
% \end{lemma}

% \begin{proof}
%   Suppose that $a\cdot (\beta-\beta')$ for $\beta,\beta'\in I$. Then
%   we have
%   \begin{align*}
%     a_1 \left((\beta)_1 - (\beta')_1\right)= \sum_{k=2}^n a_k
%     \left((\beta')_k-(\beta)_k\right) \coloneqq \hat{a}\cdot \left(\hat{\beta'}-\hat{\beta'}\right)
%   \end{align*}
%   for $\hat{a} = (a_2,\cdots , a_n)$ and $\hat{\beta}= (\beta_2,\cdots
%   , \beta_n),\hat{\beta'}= (\beta_2',\cdots , \beta_n')$ being the
%   original vectors without the first component in $\field{R}^{n-1}$
%   and $\naturals_0^{n-1}$ respectively.

%   Now we assume that $\left((\beta)_1 - (\beta')_1\right)\geq 0$,
%   otherwise just switch the two multiindices. If $\left((\beta)_1 -
%     (\beta')_1\right)$ is not zero we get the following by our assumption on the entries of $a$:
%   \begin{align*}
%     a_1 \left((\beta)_1 - (\beta')_1\right) \geq a_1 > \sum_{k=2}^n
%     M_k a_k \geq \sum_{k=2}^n \abs{(\beta')_k - (\beta)_k} a_k \geq
%     \sum_{k=2}^n \left((\beta)_1 - (\beta')_1\right) a_k.
%   \end{align*}
%   This strict inequality is a contradiction, yielding $(\beta)_1 =
%   (\beta')_1$ since $a_1 \neq 0$. Now we can consider the vector
%   $\hat{a}$ with multiindices $\hat{I}= \{\hat{\beta}_i\}_{i =
%     1,\cdots ,m }\subset \naturals_0^{n-1}$ and proceed inductively,
%   since $\hat{a}$ fulfills an analogous condition for the multiindices
%   $\{\hat{\beta}_i\}_{i=1,\cdot , m}\subset \naturals_0^{n-1}$. Thus $(\beta)_k =
%   (\beta')_k$ for $k = 1,\cdots ,n$ follows.
% \end{proof}


% \begin{definition}
%   \label{def:algebraically-independent}
%   Consider a $\field{k}$-algebra $\algebra{A}$. We call a subset of
%   elements $\{x_1,\cdots, x_n\}\subset \algebra{A}$ algebraically
%   independent if there exists no non-zero polynomial $p\in
%   \field{k}[t_1,\cdots, t_n]$ such that $p(x_1,\cdots, x_n) = 0$ in
%   $\algebra{A}$ holds.

%   In this case $\field{k}\langle x_1,\cdots, x_n\rangle \cong
%   \field{k}[t_1,\cdots,t_n]$, i.e.\ the subalgebra generated by the
%   $x_i$ is isomorphic to a polynomial algebra.
% \end{definition}


% \begin{theorem}[Noether normalization theorem]
%   \label{thm:noether-normalization}
%   Let $\field{k}$ be a field and let $\algebra{A}=\field{k}\lange
%   x_1,\cdots , x_n\rangle$ be a finitely
%   generated $\field{k}$ algebra. Then there exists an injective
%   $\field{k}$-algebra homomorphism $\field{k}[z_1,\cdots
%   ,z_r]\rightarrow \algebra{a}$ from a polynomial ring over
%   $\field{k}$ that $\algebra{A}$ into a finite extension ring of
%   $\field{k}[z_1,\cdots,z_r]$.


%   Moreover, if $\field{k}$ is infinite, the images of $z_1,\cdots
%   ,z_n$ in $\algebra{A}$ can be chosen to be $\field{k}$-linear
%   combinations of the generators of $\algebra{A}$.
% \end{theorem}

% \begin{proof}
%   We will prove the statement by induction on the number $n$ of
%   generators of $\algebra{A}$. The case $n = 0$ is trivial, as we can
%   then choose $r = 0$ as well. So assume now that $n > 0$. We have to
%   distinguish two cases:

%   \begin{enumerate}
%   \item[(a)] Assume that $\{x_1,\cdots ,x_n\}$ is algebraically
%     independent. Then, since these generate $\algebra{A}$ we can
%     choose $r = n$ and the isomorphism $\field{k}[t_1,\cdots, t_n] \rightarrow
%     \algebra{A}$ given by $t_i \mapsto x_i$ for all $i$, see \autoref{def:algebraically-independent}.
%   \item[(b)] Since $\{x_1,\cdots ,x_n\}$ are not algebraically
%     independent, we can choose a polynomial $f\in
%     \field{k}[t_1,\cdots, t_n]$ such that $f(x_1,\cdots ,x_n)= 0$ in
%     $\algebra{A}$. 
%   \end{enumerate}
% \end{proof}

%%% Local Variables:
%%% mode: latex
%%% TeX-master: "main"
%%% End:
